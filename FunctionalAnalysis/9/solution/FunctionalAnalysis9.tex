%!Mode:: "TeX:UTF-8"
%!TEX encoding = UTF-8 Unicode
%!TEX TS-program = xelatex
\documentclass{ctexart}
\newif\ifpreface
\prefacetrue
\input{../../../global/all}
\begin{document}
\large
\setlength{\baselineskip}{1.2em}
\ifpreface
    \input{../../../global/preface}
\else
\maketitle
\fi
\newgeometry{left=2cm,right=2cm,top=2cm,bottom=2cm}
%from_here_to_type
\begin{problem}
    $(C[0,1],\norm{\cdot}_1)$, let $f:C[0,1]\to \mathbb{R}$, $x\mapsto\int_0^1sx(s)\d s$. 
    Prove $f$ is continous linear functional on $C[0,1]$, calculate $\norm{f}$.
     
\end{problem}
\begin{solution}
  \begin{enumerate}
    \item \(f\) is continous linear functional on \(C[0,1]\):
      \(\forall a,b \in \mathbb{R}, \forall x,y \in C[0,1], f(ax+by)=\int_{0}^{1} s(ax(s)+by(s)) \d s=a\int_{0}^{1} sx(s)\d s+b\int_{0}^{1}  sy(s)\d s=af(x)+bf(y)\), 
      \(|f(x)-f(y)| = |\int_{0}^{1} sx(s) \d s- \int_{0}^{1} sy(s)\d s|= |\int_{0}^{1} s(x(s)-y(s)) \d s|\leq \int_{0}^{1} |x(s)-y(s)|\d s \leq \int_{0}^{1} \norm{x-y} \d s=\norm{x-y}\). 
      So \(f\) is continous linear functional.
    \item \(\norm{f}=\sup_{\norm{x}=1} |\int_{0}^{1} sx(s)|\d s \leq \sup_{\norm{x} = 1 } \int_{0}^{1} |x(s)| \d s = 1\).
      Let \(x_n=(n+1)s^n\) and \(\norm{x}=1\), then \(\norm{f(x)}=\int_{0}^{1} (n+1)s^{n+1} \d s = \frac{n+1}{n+2} \to 1, n \to \infty\).
      So, \(\norm{f}=1\).
  \end{enumerate}
\end{solution}

\begin{problem}
    $T:(\mathbb{R}^n,l^1)\to (\mathbb{R}^n,l^1)$ is linear operation. Calculate $\norm{T}$.
\end{problem}
\begin{solution}
  Let \(A\) be the matrix of linear operation \(T\). \(\forall x \in \mathbb{R}^{n} , let x = (x_1,\cdots,x_n), \), \(A=(a_{ij})_{n \times n}\). 
  \(\forall \norm{x}=1\), i.e. \(\sum_{i=1}^{n} |x_i|=1\):
  \begin{equation}
    \begin{aligned}
      \, \norm{Ax}=&\sum_{i=1}^{n} |\sum_{j=1}^{n} a_{ij}x_j|\\
    \,\,\,\  \leq&\sum_{i=1}^{n} \sum_{j=1}^{n} |a_{ij}||x_j|\\
    \, \leq&\sum_{j=1}^{n} |x_j|\sum_{i=1}^{n} |a_{ij}|\\
    \leq&\sup_{1 \leq j \leq n} \sum_{i=1}^{n} |a_{ij}|\\
    \end{aligned}
  \end{equation}
While \(\forall 1 \leq j \leq n,\) \(x_k=\mathbbm{1}_{k=j}, \sum_{i=1}^{n} |a_{ij}|=\sum_{i=1}^{n} |a_{ij}x_j|
=\sum_{i=1}^{n} \sum_{j=1}^{n} |a_{ij}x_j|=\norm{Ax}\), 
so \(\sup_{1 \leq j \leq n} \sum_{i=1}^{n} |a_{ij}|\leq \norm{Ax}\).
  
\end{solution}

\begin{problem}
    $f: C[a,b]\to \mathbb{R}$, $x\mapsto x(a)-x(b)$. Prove $f$ is bounded linear functional, calculate $\norm{f}$.
\end{problem}
\begin{solution}
  \begin{enumerate}
    \item \(f\) is bounded linear functional:
      \(\forall x \in C[0,1],\norm{x}=1, |x(a)-x(b)| \leq 2 \max_{0 \leq t \leq 1} |x(t)|=2 \) 
      \(\forall x,y \in C[0,1], k,s \in \mathbb{R}, f(kx+sy) = kx(a)+sy(a)-kx(b)-sy(b)=k(x(a)-x(b))+s(y(a)-y(b))=kf(x)+sf(y)\)
      \(x=\frac{2}{b-a}(t-a)-1 \in C[a,b]\), and \(| f(x)|=|x(a)-x(b)|=2, \norm{x} = 1\). So \(\norm{f}=2\)
  \end{enumerate}
\end{solution}


\begin{problem}
    $f: \mathcal{X}\to \mathbb{R}$,$x\mapsto\int_0^1\sqrt{t} x(t^2)\d t$.Calculate $\norm{f}$
    \begin{enumerate}
        \item $\mathcal{X}=C[0,1]$.
        \item \(\mathcal{X} = L^{ 2}[0,1]\)
    \end{enumerate}
\end{problem}
\begin{solution}
   \(\int_{0}^{1} \sqrt{t} x(t^2) \d t=\int_{0}^{1} \frac{1}{2u^{\frac{1}{4}}} x(u) \d u  \) 
  \begin{enumerate}
    \item \(\forall x \in \mathcal{X}, \norm{x} = 1, |f(x)| \leq | \int_{0}^{1} \frac{1}{2u^{\frac{1}{4}}} x(u) \d u| \leq \int_{0}^{1} |\frac{1}{2u^{\frac{1}{4}}} x(u)| \d u \leq \int_{0}^{1} \frac{1}{2u^{\frac{1}{4}}} \d u= \frac{2}{3}\).
      \(x=1 \in C[0,1]\) and \(|f(x)|=\frac{2}{3}\). So \(\norm{f}=\frac{2}{3}\).
    \item \(\forall x \in \mathcal{X} , \norm{x}=1, |f(x)|=\frac{1}{2} |\int_{0}^{1} \frac{1}{u^{\frac{1}{4}}} x(u) \d u|
      \leq \frac{1}{2} \int_{0}^{1} |\frac{1}{u^{\frac{1}{4}}}x(u)| \d u
      \leq \frac{1}{2} (\int_{0}^{1} (\frac{1}{u^{\frac{1}{4}}})^2 \d u)^{\frac{1}{2}}(\int_{0}^{1} x(u)^{2} \d u)^{\frac{1}{2}} 
      = \frac{1}{2} (\int_{0}^{1} \frac{1}{u^{\frac{1}{2}}} \d u)^{\frac{1}{2}} = \frac{\sqrt{2}}{2}\)
      Let \(x=a \frac{1}{u^{\frac{1}{4}}}\) ,then \(\int_{0}^{1} a^{2} \frac{1}{u^{\frac{1}{2}}} \d u = 1\), so \(a= \pm \frac{\sqrt{2}}{2}\).
      So \(\norm{f}=\frac{\sqrt{2}}{2}\).
  \end{enumerate}
\end{solution} 

\begin{problem}
    $\Phi: C[0,1]\to \mathbb{R}$, $\Phi(f)\mapsto \int_0^1\phi(t)f(t)\d t$, where $\phi\in C[0,1]$ Calculate $\norm{\Phi}$
\end{problem}
\begin{solution}
  \begin{enumerate}
    \item \(\Phi\) is well-defined: Obviously.
    \item \(\Phi\) is linear: Obviously.
    \item \(|\int_{0}^1 \phi(t)f(t) \d t| \leq \int_0^1 |\phi(t)f(t)| \d t \leq \int_{0}^1 |f(t)| \d t \norm{\phi}  \).
      So \(\norm{\Phi} \leq \int_{0}^1 |f(t)| \d t\).
      Let \(g(t) = \mathrm{Sgn}(f(t))\), so \(g\) is measurable. 
      And \(\int_{0}^1 g(t)f(t) \d t = \int_{0}^1 |f(t)| \d t\). By Lusin theorem, 
      \(\forall \varepsilon>0\), \(\exists h \in C[0,1]\), such that \(A=\{x \in [0,1]: h(x) \neq g(x), |h(x)| \leq 1\}, m(A) < \varepsilon\) and \(h(1) = 1\).
      So \(|\int_{0}^1 g(t)f(t)- h(t)f(t) \d t| \leq \int_{0}^1 |g(t)-h(t)| |f(t)| \d t \leq 2\varepsilon \int_{0}^1 |f(t)| \d t \to 0, \varepsilon \to 0  \).
      Therefore, \(\norm{\Phi} = \int_{0}^1 |f(t)| \d t \).
  \end{enumerate}
\end{solution}





\end{document}
