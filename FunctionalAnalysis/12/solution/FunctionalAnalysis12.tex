%!Mode:: "TeX:UTF-8"
%!TEX encoding = UTF-8 Unicode
%!TEX TS-program = xelatex
\documentclass{ctexart}
\newif\ifpreface
\prefacetrue
\input{../../../global/all}
\begin{document}
\large
\setlength{\baselineskip}{1.2em}
\ifpreface
    \input{../../../global/preface}
\else
\maketitle
\fi
\newgeometry{left=2cm,right=2cm,top=2cm,bottom=2cm}
%from_here_to_type
\begin{problem}\label{pro:1}
\(\mathcal{X}\) is a linear space on \(\mathbb{C}\). $p$ is a seminorm on \(\mathcal{X}\). \(p(x_0) \neq 0, x_0 \in \mathcal{X}\).
Prove: \(\exists f\) is a linear funcional on \(\mathcal{X}\) such that
\begin{enumerate}
  \item \(f(x_0)=1\)
  \item \(|f(x_0)| \leq \frac{p(x)}{p(x_0)}\), \(\forall x \in \mathcal{X}\).
\end{enumerate}
\end{problem}
\begin{solution}
  Consider \(p^*: \mathcal{X} \to \mathbb{C}\), \(x \mapsto \frac{p(x)}{p(x_0)}\). Obviously \(p^*\) 
  is a seminorm on \(\mathcal{X}\). Let \(\mathcal{X}_0:=\Span\{x_0\} \subset \mathcal{X}\)
  is a subspace of \(\mathcal{X}\). \(f: \mathcal{X}_0 \to \mathbb{C}\), \(\alpha x_0 \mapsto \alpha f(x_0)\),
  where \(f(x_0)=1\). So \(f\) is a linear funcional on \(\mathcal{X}_0\). And \( \forall x \in \mathcal{X}_0, x= \alpha x_0, |f(x)|=|\alpha||f(x_0)| \leq |\alpha| = \frac{p(\alpha x_0)}{p(x_0)}=p^*(x)\).
  Thus, by Hahn-Banach theorem, \(\exists \tilde{f}: \mathcal{X} \to \mathbb{R}\) is a linear funcional on \(\mathcal{X}\) such that
  \begin{enumerate}
    \item \(\tilde{f}(x)=f(x), \forall x \in \mathcal{X}_0\).
    \item \(|\tilde{f}(x)| \leq p^*(x)=\frac{p(x)}{p(x_0)}, \forall x \in \mathcal{X}\).
  \end{enumerate}
  So \(\tilde{f}(x_0)=f(x_0)\). 
\end{solution}

\begin{problem}
\(\mathcal{X}\) is a \(B^*\) space, \(\{x_n\}_{n=1}^{\infty} \subset \mathcal{X}\) such that \(\forall f \in \mathcal{X}^*, \{f(x_n)\}_{n=1}^{\infty}\) is bounded.
Prove that \(\{x_n\}_{n=1}^{\infty}\) is bounded.
\end{problem}
\begin{solution}
  Since there is an embedding map from \(\mathcal{X} \to {\mathcal{X}^*}^*\), which keeps norm. Regard \(\{x_n\}_{n=1}^{\infty}\) as subset of \({\mathcal{X}^*}^*\).
  And \({\mathcal{X}^*}^*= \mathcal{L}(\mathcal{X}^*, \mathbb{K})\), \(\mathcal{X}^*=\mathcal{L}(X, \mathbb{K})\). 
  \(\mathbb{K}\) is complete, so \(\mathcal{X}^*\) is a \(B\) space.
  Besides, \(\forall f \in \mathcal{X}^*, \sup_{n \in \mathbb{N}_+} |x_n(f)|=\sup_{n \in \mathbb{N}_+} |f(x_n)|<\infty\).
  By Banach-Steinhaus theorem, \(\sup_{n \in \mathbb{N}_+}\norm{x_n}<\infty \)
\end{solution}
\begin{problem}
\(\mathcal{X}\) is a \(B^*\) space, \(\mathcal{X}_0\) is a closed subspace of \(\mathcal{X}\).
Prove that \(\forall x \in \mathcal{X}, \inf_{y \in \mathcal{X}_0}\norm{x-y}=\sup\{|f(x)|:f \in \mathcal{X} \norm{f}=1, f|_{\mathcal{X}_0}=1\} \).

\end{problem}
\begin{lemma}\label{lem:1}
  \(\mathcal{X}\) is a \(B^*\) space, let \(H_{f}^{\lambda}:=\{x \in \mathcal{Z}: f(x)=\lambda\}\) where is a linear functional on \(\mathcal{X}\). 
  If \(\norm{f}=1\), then \(|f(x)| =  d(x, H_f^0), \forall x \in \mathcal{X}\), where \(\mathrm{d}(x, H_f^0):=\inf_{z \in H_f^0}\norm{x-z}\). 

\end{lemma}
\begin{proof}
  Since \(\norm{f}=1\), then \(\exists z \notin H_f^0\). And \(x \in H_f^0\), \(f(x)=0\), let \(z=x\), then \(\norm{x-z}=0\). Next consider \(x \notin H_f^0\):
  \begin{enumerate} 
    \item \(|f(x)| \leq \mathrm{d}(x, H_f^0)\): Since \(\forall \varepsilon >0\), \(\exists y \in H_f^0\)
      such that \(\norm{x-y} \leq \mathrm{d}(x, H_f^0)+ \varepsilon \). And \(|f(x)|=|f(x-y)| \leq \norm{f}\norm{x-y} \leq \mathrm{d}(x,H_f^0)+ \varepsilon \to \mathrm{d}(x,H_f^0), \varepsilon \to 0\).
    \item \(|f(x) \geq d(x, H_f^0)|\): \(\forall \varepsilon > 0 \), \(\exists y \in x\) 


  \end{enumerate}
\end{proof}


\begin{solution}
 By \Cref{lem:1}, we have that \(\inf_{y \in \mathcal{X}_0}\norm{x-y} \geq \sup\{|f(x)|:f \in \mathcal{X}^*, \norm{f}=1, f|_{\mathcal{X}_0}=0\} \).
 Consider \(\mathcal{X}_0=\mathcal{X}\) is possible, we define: \(\sup \varnothing =0\).
 \begin{enumerate}
   \item \(\mathcal{X}_0=\mathcal{X}\): \(f(x)=0, \forall x \in \mathcal{X}\), so \(\notexists f: \norm{f}=1\), so \(\sup \{|f(x)|: f \in \mathcal{X}^*, \norm{f}=1, f|_{\mathcal{X}_0}=0\}= \sup \varnothing\).
     Obviously, the conclusion is true.
   \item \(\mathcal{X}_0 \subsetneq \mathcal{X}\):\begin{enumerate}
     \item If \(x \notin \mathcal{X}_0\), since \(\mathcal{X}_0\) is closed, then
       \(d:=\inf_{y \in \mathcal{X}_0}\norm{x-y} > 0\), by Hahn-Banach theorem, \(\exists f \in \mathcal{X}^*\) such that
       \(\norm{f}=1, f|_{\mathcal{X}_0}=0, f(x)=d\). Thus, \( \sup\{|f(x)|:f \in \mathcal{X}^*, \norm{f}=1, f|_{\mathcal{X}_0}=0\}=|f(x)| \).
       So \(\inf_{y \in \mathcal{X}_0 \norm{x-y}}= |f(x)|= \sup\{|f(x)|:f \in \mathcal{X}^*, \norm{f}=1, f|_{\mathcal{X}_0}=0\} \).
     \item If \(y \in \mathcal{X}_0\), take \(x \notin \mathcal{X}_0\),\(f\) such that \( \sup\{|f(x)|:f \in \mathcal{X}^*, \norm{f}=1, f|_{\mathcal{X}_0}=0\}=|f(x)| \).
       So \(f(y)=0\), then  \(\{|f(y)|: f \in \mathcal{X}^*, \norm{f}=1, f|_{\mathcal{X}_0}=0\}\) is not empty.
       And \(\forall f \in \mathcal{X}^* \) such that \(\norm{f}=1, f|_{\mathcal{X}_0}=0\), then \(f(y)=0\). 
       Thus,  \( \sup\{|f(x)|:f \in \mathcal{X}^*, \norm{f}=1, f|_{\mathcal{X}_0}=0\}=0=\inf_{y \in \mathcal{X}_0}\norm{x-y} \).


   \end{enumerate}

 \end{enumerate}
\end{solution}
\end{document}
