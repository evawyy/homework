%!Mode:: "TeX:UTF-8"
%!TEX encoding = UTF-8 Unicode
%!TEX TS-program = xelatex × 2
\documentclass{ctexart}
\newif\ifpreface
\prefacetrue
\input{../../../global/all}
\begin{document}
\large
\setlength{\baselineskip}{1.2em}
\ifpreface
    \input{../../../global/preface}
\else
\maketitle
\fi
\newgeometry{left=2cm,right=2cm,top=2cm,bottom=2cm}
%from_here_to_type
\begin{problem}
    Let $A=\{e_k\}$ are orthonormal basis in inner product space $X$. Prove: $\forall x,y\in X$, $\sum_{k=1}^{\infty}|(x,e_k)(y,e_k)|\leq \norm{x}\norm{y}$.
\end{problem}
\begin{solution}
    Since $A=\{e_k\}$ are orthonormal basis in inner product space $X$, then $\forall x\in X$, $\norm{x}=\sum_{k=1}^{\infty}|(x,e_k)|^2$, so by Holder inequation, we get $\sum_{k=1}^{\infty}|(x,e_k)(y,e_k)|\leq (\sum_{k=1}^{\infty}|(x,e_k)|^2)^{\frac{1}{2}}(\sum_{k=1}^{\infty}|(y,e_k)|^2)^{\frac{1}{2}}=\norm{x}\norm{y}$.
\end{solution}

\begin{problem}
    $H$ is Hilbert space, $\{e_k\},\{e_k'\}$ are two kinds of orthonormal set in inner product space $H$, $\sum_{k=1}^{\infty}\norm{e_k-e_k'}^2<1$. Prove: if one of $\{e_k\},\{e_k'\}$ is complete, then the other is complete.
\end{problem}
\begin{solution}
    Let $\{e_n\}_{n=1}^{\infty}$ is complete. If $\{e_n'\}_{n=1}^{\infty}$ is not complete, then $\exists x_0:\ \theta\neq x_0\notin Span\{\{e_n'\}_{n=1}^{\infty}\}$ s.t. $(x_0,e_n')=0\forall n$. So $\norm{x_0}^2=\sum_{n=1}^{\infty}|(x_0,e_n)|^2=\sum_{n=1}^{\infty}|(x_0,e_n-e_n')|^2\leq\norm{x_0}^2\sum_{n=1}^{\infty}\norm{e_n-e_n'}^2<\norm{x_0}$. Contradiction!
\end{solution}

\begin{problem}
    $H$ is an inner space, these propositions below are equal:
    \begin{enumerate}
        \item $x\perp y$;
        \item $\norm{x+\alpha y}\geq \norm{x}$, $\alpha\in \mathbb{C}$;
        \item $\norm{x+\alpha y}=\norm{x-\alpha y},\forall \alpha\in \mathbb{C}$.
    \end{enumerate}
\end{problem}
\begin{solution}
    \begin{enumerate}
        \item $x\perp y \Rightarrow \norm{x+\alpha y}\geq \norm{x}$, $\alpha\in \mathbb{C}:$ $\norm{x+\alpha y}=\norm{x}+|\alpha|\norm{y}\geq \norm{x},\forall \alpha\in \mathbb{C}$.
        \item $x\perp y \Leftarrow \norm{x+\alpha y}\geq \norm{x}$, $\alpha\in \mathbb{C}:$ If $(x,y)\neq 0$, let $\alpha=r(x,y),r\in \mathbb{R}$, then by $\norm{x+\alpha y}\geq \norm{x}$, we get $|\alpha|^2\norm{y}^2+2\re\overline{\alpha}(x,y)\geq 0$, that is $r^2|(x,y)|^2\norm{y}^2+2r|(x,y)|^2\geq 0$, so $\Delta=4\leq 0$. Contradiction!
        \item $x\perp y \Rightarrow \norm{x+\alpha y}=\norm{x-\alpha y},\forall \alpha\in \mathbb{C}$: $\norm{x+\alpha y}=\norm{x}+|\alpha|\norm{y}=\norm{x-\alpha y}$.
        \item $x\perp y \Leftarrow \norm{x+\alpha y}=\norm{x-\alpha y},\forall \alpha\in \mathbb{C}$: If $(x,y)\neq 0$, let $\alpha=r(x,y),r\in \mathbb{R}$, then by $\norm{x+\alpha y}= \norm{x-\alpha y}$, we get $\re\overline{\alpha}(x,y)=0$, that is $r^2|(x,y)|^2= 0,\forall r\in \mathbb{R}$. Contradiction!
    \end{enumerate}
\end{solution}

\iffalse
\begin{problem}
    $X$ is a linear space over field $\mathbb{C}$ or $\mathbb{R}$, $(X,\norm{\cdot})$ is a $B^*$ space, then $\norm{\cdot}$ can be induced by an inner product if and only if $\forall x,y\in X$, $\norm{x+y}^2+\norm{x-y}^2=2(\norm{x}^2+\norm{y}^2)$.
\end{problem}
\begin{lemma}\label{lem:1}
    $f:\mathbb{R}\to \mathbb{R}$ s.t. $f$ is continous and $f(m+n)=f(m)+f(n),\forall m,n\in \mathbb{R}$, then $f(x)=xf(1),\forall x\in \mathbb{R}$.
\end{lemma}
\begin{solution}
    \begin{enumerate}
        \item $\forall x\in \mathbb{N}$, apply MI to prove $f(x)=xf(1)$:
        \begin{enumerate}
            \item $f(0+0)=f(0)=f(0)+f(0)=2f(0)$, then $f(0)=0$. 
            \item If $f(x)=kf(1)$, then $f(k+1)=f(k)+f(1)=kf(1)+f(1)=(k+1)f(1)$
        \end{enumerate}
        \item $\forall -x\in \mathbb{N}_+ $, $f(-x)=f(-x)+f(x)-f(x)=f(x+(-x))-f(x)=0-xf(1)$.
        \item $\forall x\in \mathbb{Q}$, let $x=\frac{p}{q}$, where $q\neq 0, p,q\in \mathbb{Z}$, then $qf(\frac{p}{q})=f(q\frac{p}{q})=f(p)=pf(1)$, then $f(\frac{p}{q})=\frac{p}{q}f(1)$.
        \item $\forall x\in \mathbb{R}$, $\exists \{a_n\}_{n=1}^{\infty}\subset \mathbb{Q} $ s.t. $a_n\to x$, then $f(x)=f(\lim_{n\to\infty}a_n)=\lim_{n\to\infty}f(a_n)=\lim_{n\to\infty}a_nf(1)=xf(1)$.
            
        
    \end{enumerate}
\end{solution}


\begin{solution}
    ``$\Rightarrow$'': If $\norm{\cdot}$ can be induced by $(\cdot,\cdot)$, that means $\norm{x}=(x,x)^{\frac{1}{2}}$, then $\forall x, y$, $\norm{x+y}^2+\norm{x-y}^2=(x+y,x+y)+(x-y,x-y)=(x,x)+x(y,y)+2\real(x,y)+(x,x)+(y,y)-2\real(x,y)=2((x,x)+(y,y))=2(\norm{x}^2+\norm{y}^2)$.\\
    ``$\Leftarrow$'': \begin{enumerate}
        \item\label{it:01} For $\mathbb{R}$: Let $(x,y)=\frac{1}{4}(\norm{x+y}^2-\norm{x-y}^2)$, next we will prove $(x,y)$ is an innear product and $(x,x)=\norm{x}^2$:
        \begin{enumerate}
            \item\label{it:1} $(x,x)=\norm{x}^2, \forall x\in X$: $(x,x)=\frac{1}{4}(\norm{x+x}^2-\norm{x-x}^2)=\norm{x}^2$.
            \item\label{it:2} $(x,y)=(y,x)$: $(x,y)=\frac{1}{4}(\norm{x+y}^2-\norm{x-y}^2)=(y,x)$.
            \item\label{it:3} $(x+y,z)=(x,z)+(y,z),\forall x,y,z\in X$:
            \begin{equation}
                \begin{aligned}
                    (x+y,z)=&\frac{1}{4}(\norm{x+y+z}^2-\norm{x+y-z}^2)\\
                    =&\frac{1}{4}(2\norm{x}^2+2\norm{z+y}^2-\norm{x-z-y}^2-\norm{x+y-z}^2)\\
                    =&\frac{1}{4}(2\norm{x}^2+2\norm{z+y}^2-2\norm{x-z}^2-2\norm{y}^2)\\
                    =&\frac{1}{4}(2\norm{y}^2+2\norm{z+x}^2-2\norm{y-z}^2-2\norm{x}^2)\\
                    =&\frac{1}{4}\times\frac{1}{2}(2\norm{x}^2+2\norm{z+y}^2-2\norm{x-z}^2-2\norm{y}^2+2\norm{y}^2+2\norm{z+x}^2\\
                    &-2\norm{y-z}^2-2\norm{x}^2)\\
                    =&\frac{1}{4}(\norm{z+y}^2-\norm{y-z}^2+\norm{z+x}^2-\norm{x-z}^2)
                \end{aligned}
            \end{equation}
            \item\label{it:4} $(x,y+z)=(y+z,x),\forall x,y,z\in X$: $(y+z,x)=(y,x)+(z,x)=(x,y)+(x,z)=(x,y+z)$.
            \item\label{it:5} $(ax,y)=a(x,y),\forall x,y\in X,\forall a\in R$:\\
            $f:\mathbb{R}\to \mathbb{R}$, $a\mapsto (ax,y)$. $\forall a\in \mathbb{R}$, $a_n\to a$, then 
            \iffalse
            \begin{equation}
                \begin{aligned}
                &\lim_{n\to\infty}f(a_n)\\
                =&\lim_{n\to\infty}(a_nx,y)\\
                =&\lim_{n\to\infty}\frac{1}{4}(\norm{a_nx+y}^2-\norm{a_nx-y}^2)\\
                =&\frac{1}{4}(\lim_{n\to\infty}\norm{a_nx+y}^2-\lim_{n\to\infty}\norm{a_nx-y}^2)\\
                =&\frac{1}{4}(\norm{ax+y}^2-\norm{ax-y}^2)\\
                =&(ax,y)\\
                =&f(a)
                \end{aligned}
            \end{equation}
            \fi
            $f$ is obviously continous. And $f(m+n)=((m+n)x,y)=(mx+nx,y)=(mx,y)+(nx,y)=f(m)+f(n)$. Then, by \Cref{lem:1}, $f(a)=af(1)$, i.e. $(ax,y)=a(x,y)$.
            \item $0=(x,x)=\norm{x}^2$ $\iff$ $\norm{x}=0$ $\iff$ $x=0$.

        \end{enumerate}
        \item For $\mathbb{C}$: $X$ is a linear space on $\mathbb{C}$, so $X$ is a linear space on $\mathbb{R}$. So by \Cref{it:01}, $(\cdot,\cdot)_1$ defined as \Cref{it:01} is the inner product on $X$ in field $\mathbb{R}$. Let $(a,b)=(a,b)_1+\mathrm{i}(a,\mathrm{i}b)_1$, next we will prove $(X,(\cdot,\cdot))$ is an inner product space on field $\mathbb{C}$.
        \begin{enumerate}
            \item $(x,\mathrm{i} x)_1=\frac{1}{4}(\norm{x+\mathrm{i}x}^2-\norm{x-\mathrm{i}x}^2)=(x,y)=\frac{1}{4}(\norm{-\mathrm{i}x+x}^2-\norm{x-\mathrm{i}x}^2)=0$. So $(x,x)=(x,x)_1+\mathrm{i}(x,\mathrm{i}x)_1=\norm{x}^2$.
            \item Since $(\cdot,\cdot)_1$ is an inner product, so $(x+y,z)=(x,z)+(y,z)$.
            \item $\forall x,y\in X $, $(x,\mathrm{i}y)_1=\frac{1}{4}(\norm{x+\mathrm{i}y}^2-\norm{x-\mathrm{i}y}^2)=\frac{1}{4}(\norm{-\mathrm{i}x+y}^2-\norm{\mathrm{i}x+y}^2)=(y,\mathrm{i}x)_1$. So $(x,y)=(x,y)_1+\mathrm{i}(x,iy)_1=(y,x)_1+\mathrm{i}(y,\mathrm{i}x)_1=\overline{(y,x)}$. 
            \item $\forall x,y\in X $, $(x,\mathrm{i}y)=(x, \mathrm{i}y)_1-\mathrm{i}(x,y)_1=-\mathrm{i}(x,y)$. So $\forall a,b\in \mathbb{C}, k_1=\re k,k_2=\im k,k=a,b$, $(x,ay+bz)=(x,a_1y+b_1z+\mathrm{i}(a_2y+a_2z))=(x,a_1y+b_1z)-\mathrm{i}(x,a_2y+a_2z)=(x,ay)+(x,bz)$.
        \end{enumerate} 
    \end{enumerate}
\end{solution}
\fi


\end{document}