%!Mode:: "TeX:UTF-8"
%!TEX encoding = UTF-8 Unicode
%!TEX TS-program = xelatex
\documentclass{ctexart}
\newif\ifpreface
\prefacetrue
\input{../../../global/all}
\begin{document}
\large
\setlength{\baselineskip}{1.2em}
\ifpreface
    \input{../../../global/preface}
\else
\maketitle
\fi
\newgeometry
%from_here_to_type
\begin{problem}
  Consider \(\mathbb{Q} = \mathbb{Z} \times (\mathbb{Z} \setminus \{0\})/ \sim \), where \( (a,b) \sim (c,d) \iff ad=bc\).
  Define \( +_ \mathbb{Q}, \cdot_{\mathbb{Q}}\) and \(<_{\mathbb{Q}}\) and verifty that your definitions
  don't depend on the choice of representatives.
\end{problem}
\begin{solution}
	Define \([(a,b)]+_\mathbb{Q}[(c,d)]=[(ad+bc,bd)],[(a,b)]\cdot_\mathbb{Q}[(c,d)]=[(ac,bd)]\),
	and \([(a,b)]<_\mathbb{Q} [(c,d)]\iff a b d^2<c d b^2\).
	Next to prove these definitions don't depend on the choice of representatives.
  \begin{enumerate}
    \item \(+_\mathbb{Q}\): Let \((a,b)\sim (e,f), (c,d) \in \mathbb{Q})\), so \(af=be\). 
      Thus, \((ad+bc)bf=ad^2f+bdcf=bed^2+bdcf=(ed+fc)bd\).
      So \((ad+bc,bd)\sim (ed+fc,df)\). So \(+_\mathbb{Q}\) is well defined.
    \item \(\cdot_\mathbb{Q}\): Let \((a,b)\sim (e,f), (c,d) \in \mathbb{Q})\), so \(af=be\).  
      Then, \(acfd=bced=bdec\), i.e. 
	    \((ac,bd)\sim (ec,fd)\).
    \item \(<_\mathbb{Q}\): Let \((a_1,b_1)\sim (a_2,b_2),(c_1,d_1)\sim(c_2,d_2)\)
      and \((a_1,b_1)<(c_1,d_1)\). So \(a_1b_2=a_2b_1,c_1d_2=c_2d_1\) 
      and	\(a_1 b_1 d_1^2<c_1 d_1 b_1^2\). Thus, \(a_1 b_1 d_1^2 b_2^2 d_2^2 < c_1 d_1 b_1^2 b_2^2 d_2^2\).
      Then, \(a_2 b_1^2 d_1^2 b_2 d_2^2 < c_2 d_1^2 b_1^2 b_2^2 d_2\).
      So \(a_2 d_2^2 b_2 < c_2 b_2^2 d_2\). Therefore, we prove \((a_2,b_2)<(c_2,d_2)\).
  \end{enumerate}
\end{solution}
\end{problem}


\begin{problem}
The set of all continuous functions $f: \mathbb{R} \rightarrow \mathbb{R}$ has cardinality $\mathfrak{c}$
(while the set of all functions has cardinality $2^{\mathfrak{c}}$ ).
[A continuous function on $\mathbb{R}$ is determined by its values at rational points.]
\end{problem}

\begin{solution}
  Let \(S:=\{f \in \fun{\mathbb{R}}{\mathbb{R}}: f is continous\}\).
  Consider \(\theta: S \to 2^{\mathbb{Q}}, f \mapsto \{(a,b) \in \mathbb{Q} \times \mathbb{Q}: f(a) < b\}\). 
  \begin{enumerate}
    \item \(f\) is a injection: Assume \(\theta(f)=\theta(g)\),
      \begin{enumerate}
        \item \(\forall x\in \mathbb{Q}\), so
          \(f(x)=\sup \{y \in \mathbb{Q}:y<f(x)\}=\sup\{y \in \mathbb{Q}:(x,y) \in \theta(f)\}
          =\sup\{y \in \mathbb{Q} : (x,y) \in \theta (g)\}\sup {y \in \mathbb{Q}: y < g(x)} =g(x)\).
        \item \(\forall x \in \mathbb{R}\), \(\exists \{x_n\}_{n=1}^{\infty} \subset \mathbb{Q}\) 
          such that \(x_n \to x\), then \(f(x)= \lim_{n \to \infty }f(x_n)=\lim_{n \to \infty }g(x_n)=g(x)\). 
      \end{enumerate}
      So we get \(f=g\). So \(\card \fun{\mathbb{R}}{\mathbb{R}} \leq \card 2^{\mathbb{Q}}=2^{\aleph_0}\).
     
    \item Obviously \(\card\fun{\mathbb{R}}{\mathbb{R}} \geq 2^{\aleph_0}\), so we get they are equal. 
  \end{enumerate}
\end{solution}


\begin{problem}
There are at least $\mathfrak{c}$ countable order-types of linearly ordered sets.
\end{problem}

\begin{solution}
For every sequence $a=\left\langle a_n: n \in \mathbb{N}\right\rangle$ of natural numbers consider the ordertype
$$
\tau_a=\{(x,y) \in \mathbb{Z}\times \mathbb{N}:2 \nmid y \wedge 0<x<a_{\frac{y}{2}}\}
$$
And for \((x,y),(z,w) \in \tau_a\) we define \((x,y)<(z,w) \iff y<w \wedge y=w,x<z\).
Now we will show that if $a \neq b$, then $\tau_a \neq \tau_b$. 
Assume \(\tau_a \cong \tau_b\), we need to prove \(a=b\). assume \(\theta:\tau_a \to \tau_b\) is the isomorfism. 

We know \((x,0)\) can be defined as \(\phi(p)=\exists_{k=1}^{x-1} t_k,\wedge_{1 \leq i < j \leq x-1}t_i \neq t_j, \forall k=1,\cdots x-1, t_k<p\). 
And \(\theta\) is isomorphism. So \(\theta(x,0)=(x,0)\). 
For \((x,1)\), we let \(b_0\) satisfy \(\theta(0,1)=(b_0,m)\). 
Since the set \(\{(x,y):y=1\}\) can be defined by \(\psi(p)=\forall r,s(r,s<p \wedge \tau(r)\wedge \tau(s) \to \card[r,s] < \infty )\), 
where \(\tau(r):= \{s:s<r\}\) and \([r,s]=\{y:r<y<s\}\). 
we get \(\theta[\{(x,y):y=1\}]=\{(x,y):y=1\}\). 
So we can delete the element whose second coordinary is \(0,1\), and \(\theta\) is isomorphism, too. 
Do this repeatedly, we get \(\theta(x,2n+1)=(x,2n+1)\). 
So \(a_n=\card\{(x,2n+1)\in \tau_a\}=\card \{(x,2n+1) \in \tau_b\}=b_n\) and thus \(a=b\).
\end{solution}
\end{document
