%!Mode:: "TeX:UTF-8"
%!TEX encoding = UTF-8 Unicode
%!TEX TS-program = xelatex × 2
\documentclass{ctexart}
\newif\ifpreface
\prefacetrue
\input{../../../global/all}
\DeclareMathOperator{\ord}{Ord}
\DeclareMathOperator{\otype}{OrderType}
\newcommand{\ini}{\mathrm{\mathop{ini}}}
\newtheorem{example}{Example}
\newcommand{\N}{\mathbb{N}}
\newcommand{\calA}{\mathcal{A}}
\newcommand{\len}{\mathop{\mathrm{len}}}
\newcommand{\Q}{\mathbb{Q}}
\crefname{enumi}{}{}
\begin{document}
\large
\setlength{\baselineskip}{1.2em}
\ifpreface
\input{../../../global/preface}
\else
\maketitle
\fi
\newgeometry{left=2cm,right=2cm,top=2cm,bottom=2cm}
%from_here_to_type
\section{Question}
\begin{problem}
Let $(U,\le),(V,\prec)$ be two well-orderings. Consider $f:=\{(x,y):x\in U\xand y\in V\xand (U_x,\le)\cong (V_y,\prec)\}$, prove $f$ is isomorphism from some initial segment of $U$ to some initial segment of $V$.
\end{problem}

\begin{problem}\label{pro:2}
The relation ``$(P,\le)\cong(Q,\le)$'' is an equivalence relation (on the class of all partially ordered sets).
\end{problem}



\begin{problem}
Let $\mathcal{A}$ denote the class of all well orderings. For any $a,b\in\mathcal{A}$, define $a\prec b\iff a$ is isomorphic to an initial segment of $b$. Show that $\prec$ is a well ordering on $\mathcal{A}/\cong$, where $\cong$ is the equivalence relation given in \Cref{pro:2}.
\end{problem}



\begin{problem}
\begin{enumerate}
\item If $(W,<)$ is a well ordering and $U \subset W$, then $(U,<\cap(U \times U))$ is a well ordering.
\item
If $\left(W_1,<_1\right)$ and $\left(W_2,<_2\right)$ are two well orderings and $W_1 \cap W_2=\varnothing$, then $W_1 \oplus W_2=\left(W_1 \cup W_2, \prec\right)$ is a well ordering, where
$$
\prec=<_1 \cup<_2 \cup\left\{(a, b) \mid a \in W_1 \wedge b \in W_2\right\}
$$
\item
If $\left(W_1,<_1\right)$ and $\left(W_2,<_2\right)$ are two well orderings, then $W_1 \otimes W_2=\left(W_1 \times W_2, \prec\right)$ is a well ordering, where
$$
\left(a_1, b_1\right) \prec\left(a_2, b_2\right) \leftrightarrow b_1<_2 b_2 \vee\left(b_1=b_2 \wedge a_1<_1 a_2\right)
$$
\end{enumerate}
\end{problem}


\begin{problem}
Show that the following are equivalent:
\begin{enumerate}[ref=\theproblem.\arabic*]
\item\label{it:1} $T$ is transitive;
\item\label{it:2}$\bigcup T \subseteq T$;
\item\label{it:3} $T \subseteq \mathscr{P}(T)$.
\end{enumerate}
\end{problem}



\begin{problem}
Let $\alpha, \beta, \gamma \in\ord$ and let $\alpha<\beta$. Then
\begin{enumerate}[label=\alph*,ref=\theproblem.\alph*]
\item\label{it:11} $\alpha+\gamma \leq \beta+\gamma$.
\item\label{it:12} $\alpha \cdot \gamma \leq \beta \cdot \gamma$.
\item\label{it:13} $\alpha^\gamma \leq \beta^\gamma$.
\end{enumerate}
Given examples to show that $\leq$ cannot be replaced by $<$ in either inequality.
\end{problem}




\begin{problem}
Show that the following rules do not hold for all $\alpha, \beta, \gamma \in\ord$:
\begin{enumerate}[label=\alph*,ref=\theproblem.\alph*]
\item If $\alpha+\gamma=\beta+\gamma$ then $\alpha=\beta$.
\item If $\gamma>0$ and $\alpha \cdot \gamma=\beta \cdot \gamma$ then $\alpha=\beta$.
\item $(\beta+\gamma) \cdot \alpha=\beta \cdot \alpha+\gamma \cdot \alpha$.
\end{enumerate}
\end{problem}



\begin{problem}
Find a set $A \subset \mathbb{Q}$ such that $\left(A,<_{\mathbb{Q}}\right) \cong(\alpha, \in)$, where
\begin{enumerate}[label=\alph*,ref=\theproblem.\alph*]
\item  $\alpha=\omega+1$,
\item  $\alpha=\omega \cdot 2$,
\item  $\alpha=\omega \cdot \omega$,
\item  $\alpha=\omega^\omega$,
\item  $\alpha=\varepsilon_0$.
\item  $\alpha$ is any ordinal $<\omega_1$.
\end{enumerate}
\end{problem}



\begin{problem}
An ordinal $\alpha$ is a limit ordinal iff $\alpha=\omega \cdot \beta$ for some $\beta \in\ord$.
\end{problem}



\begin{problem}
Find the first three $\alpha>0$ s.t. $\xi+\alpha=\alpha$ for all $\xi<\alpha$.
\end{problem}



\begin{problem}
Find the least $\xi$ such that
\begin{enumerate}[label=\alph*,ref=\theproblem.\alph*]
\item  $\omega+\xi=\xi$.
\item  $\omega \cdot \xi=\xi, \xi \neq 0$.
\item  $\omega^{\xi}=\xi$.
\end{enumerate}
(Hint for (1): Consider a sequence $\left\langle\xi_n\right\rangle$ s.t. $\xi_{n+1}=\omega+\xi_n$.)
\end{problem}


\end{document}


