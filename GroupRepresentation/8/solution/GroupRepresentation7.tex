%!Mode:: "TeX:UTF-8"
%!TEX encoding = UTF-8 Unicode
%!TEX TS-program = xelatex
\documentclass{ctexart}
\newif\ifpreface
%\prefacetrue
\input{../../../global/all}
\begin{document}
\large
\setlength{\baselineskip}{1.2em}
\ifpreface
	\input{../../../global/preface}
	\newgeometry{left=2cm,right=2cm,top=2cm,bottom=2cm}
\else
	\newgeometry{left=2cm,right=2cm,top=2cm,bottom=2cm}
	\maketitle
\fi
%from_here_to_type
\begin{problem}
Find all of irreducible reperentation of \(C_4=\{e,a,a^2,a^3\}\) over \(\mathbb{C}\) by give the irreducible decomposation of it's regular reperentation.
\end{problem}

\begin{solution}
	Assume \(\phi:C_4 \to M_4(\mathbb{C})\) is the regular reperentation, and
	\[
		\phi(a)=\left(\begin{array}{cccc}
				0 & 1 & 0 & 0 \\
				0 & 0 & 1 & 0 \\
				0 & 0 & 0 & 1 \\
				1 & 0 & 0 & 0 \\
			\end{array}\right)
	\]
	Let \(V_1=\{x \in \mathbb{C}^4:x_1=x_2=x_3=x_4\}, V_2=\{x \in \mathbb{C}^4:x_1=x_3=-x_2=-x_4\},V_3=\{x \in \mathbb{C}^4:x_1=-x_3,x_2=-x_4\}\).
	Easily we get \(V_1,V_2,V_3\) are invariant subspace over \(\phi\). Now we prove thry are irreducible.
	Obviously \(\dim V_1=\dim V_2=1\), so they are irreducible. Only need to prove \(V_3\) is irreducible.
	Consider \(W \subset V_3\) is a subspace and \(W \neq \{0\}\), to prove \(W=V_3\).
	Let \(x \in W\) and \(x \neq 0\). Then \(\phi(a)x=(x_2,x_3,x_4,x_1) \in W\).
	Consider the equation \(\begin{cases}
		a x_1+b x_2=1 \\
		a x_2 - b x_1 =0
	\end{cases}\), Since \(x_1,x_2\) can't be all \(0\), we know this eauqtion has a solution \((a,b)\).
	Then \((1,0,-1,0)=ax+b \phi(x) \in W\). For the same reason we get \((0,1,0,-1) \in W\), too.
	So \(W=V_3\). So \(V_3\) is irreducible.
	Easily we find \(\phi|_{V_1}\) is ordinary reperentation,
	\(\phi|_{V_2}\) is isomorphic to \(\psi:C_4 \to \mathbb{C}, a \mapsto -1\),
	and \(\phi|_{V_3}\) is isomorphic to \(\tau:C_4 \to M_2(\mathbb{C}), a \mapsto \begin{pmatrix}
		0  & 1 \\
		-1 & 0 \\
	\end{pmatrix}\).
	They are all of irreducible reperentation of \(C_4\) over \(\mathbb{C}\).
\end{solution}
\end{document}
