\documentclass{ctexart}
%\usepackage{xeCJK}
%\usepackage[T1]{fontenc}
%\usepackage{mathptmx}
\usepackage{amsmath,amssymb,amsthm,color,mathrsfs}
\usepackage{enumitem,anysize}
\usepackage{geometry}
\usepackage{lipsum}
\usepackage{bbm}
\usepackage{tikz}
\usepackage{hyperref}
\hypersetup{
	hypertex=true,
	colorlinks=true,
	linkcolor=red,
	filecolor=blue,      
	urlcolor=blue,
	citecolor=cyan,
}

\geometry{a4paper,left=1cm,right=1cm,top=1cm,bottom=1cm}

\def\<{\langle}
\def\>{\rangle}
\edef\lim{\displaystyle\lim}
\def\email#1{\href{mailto:#1}{\texttt{#1}}}

\newtheorem{problem}{\textbf{Problem}}
\renewcommand\theproblem{\textbf{\Roman{problem}}}
\newenvironment{solution}{\begin{proof}[\textbf{Solution}]}{\end{proof}}

\renewcommand\phi{\varphi}
\renewcommand{\(}{\left(}
\renewcommand{\)}{\right)}
\renewcommand{\d}{\mathrm{d}}
\newcommand{\supp}{\mathrm{supp}}
\newcommand{\N}{\mathbb{N}}
\newcommand{\E}{\mathbb{E}}
\newcommand{\R}{\mathbb{R}}
\newcommand{\Z}{\mathbb{Z}}
\newcommand{\C}{\mathbb{C}}
\newcommand{\shi}{\mathbbm{1}}
\renewcommand{\epsilon}{\varepsilon}
\renewcommand{\phi}{\varphi}
\newcommand{\id}{\mathrm{id}}
\renewcommand{\div}{\mathrm{div}}
\renewcommand{\arg}{\mathrm{arg}}
\newcommand{\e}{\mathrm{e}}
\newcommand{\chara}{\mathrm{char}}
\newcommand{\tra}{\mathrm{tr}}
\newcommand{\GL}{\mathrm{GL}}
\newcommand{\zzh}{\textsuperscript{T}}
\newcommand{\calO}{\mathcal{O}}
%\newcommand{\deg}{\mathrm{deg}}
\newcommand{\eto}{\mapsto}




\newcommand{\minus}{\mathbin{\backslash}}

\newtheorem{lemma}{Lemma}
\newtheorem{cora}{Corallary}

\iffalse


.

\fi



\pagestyle{empty}
\title{$\mathbbm{Group\quad Representation}$}
\author{王胤雅\\
SID:201911010205\\
\email{201911010205@mail.bnu.edu.cn}}
\begin{document}
\large
\maketitle
\begin{problem}
    Group $G$ has an action on set $\Omega=\left\{x_1, x_2, \cdots, x_n\right\}$, let $(\phi, V)$ be the $n-$ dimensional $K$ permutation representation of $G$, where $K$ is the field of vector space $V$, and 
    $$
V=\left\{\sum_{i=1}^n a_i x_i \mid a_i \in K, i=1,2, \cdots, n\right\} .
$$
Let
$
 V_1=\left\langle\sum_{i=1}^n x_i\right\rangle, 
 V_2=\left\{\sum_{i=1}^n a_i x_i \mid \sum_{i=1}^n a_i=0, a_i \in K\right\} .
$
Proof: (1) $V_1$ and $V_2$ are  invariant subspaces of $G$ ;
(2) If $\chara K \nmid n$, then $\varphi=\varphi_{V_1} \oplus \varphi_{V_2}$.
\end{problem}
\begin{solution}
\begin{enumerate}
    \item $\forall \sigma\in S_n$, $\forall v_i\in V_i,i=1,2$, 
    then $v_1=k\sum_{i=1}^n x_i,k\in K,v_2=\sum_{i=1}^n a_i x_i, \sum_{i=1}^n a_i=0$, 
    so $\sigma(v_1)=k\sum_{i=1}^n x_{\sigma(i)}=k\sum_{i=1}^n x_i$, 
    $\sigma(v_2)=\sum_{i=1}^n a_i x_{\sigma(i)}=\sum_{i=1}^n a_{\sigma^{-1}(i)} x_i,$ $\sum_{i=1}^n a_{\sigma^{-1}(i)}=0$, so $\sigma(v_2)\in V_2$
    Since $\phi(G)\cong H\leq S_n$, then $\forall g\in G$, $\phi(g)(v_1)=v_1\in V_1,\phi(g)(v_2)\in V_2$.
    \item We only need to proof $V=V_1\oplus V_2$.
    $\forall v\in V_1\cap V_2$, $v=k\sum_{i=1}^n x_i=\sum_{i=1}^n a_i x_i$ where $k\in K ,\sum_{i=1}^n a_i=0$, then $\sum_{i=1}^n(k-a_i)x_i=0$, so $k-a_i=0,\forall 1\leq i\leq n$, which means $\sum_{i=1}^n a_i=nk=0$. 
    Since $\chara K\nmid n$, then $k=0$. Thus, $v=0$. $\forall u\in V$, let $k=\frac{1}{n}\sum_{i=1}^nb_i$, $u=\sum_{i=1}^n b_ix_i$, so $u=k\sum_{i=1}^nb_ix_i+\sum_{i=1}^n(b_i-k)x_i$. By noting that $\sum_{i=1}^n(b_i-k)=\sum_{i=1}^nb_i-n\times k=0$, $V=V_1+V_2$.
\end{enumerate}


\end{solution}
\begin{problem}
 Using exercise $1$, calculate a $2-$dimensional complex representation of $S_3$ and its matrix of the representation.
\end{problem}
\begin{solution}
Let $\Omega=\{x_1,x_2\}, V:=\{a_1x_1+a_2x_2:a_i\in K,i=1,2\}$, $\phi: S_3\to \GL(V)$. Since $\forall a_1x_1+a_2x_2=0, a_1+a_2=0$, then $a_1=-a_2$, $a_1x_1+a_2x_2=a_1x_1-a_1x_2$, so $V_2=\langle x_1-x_2\rangle$. 
$\forall \sigma\in S_3$, $\phi(\sigma)=\id$, when $\sigma$ is an even permutation; $\phi(\sigma)=(12)$, when $\sigma$ is an odd permutation. So $\phi(\sigma)|_{V_1}=\id,\phi(\sigma)|_{V_2}=\id$, when $\sigma$ is even; $\phi(\sigma)|_{V_1}=\id,\phi(\sigma)|_{V_2}:V_2\to V_2, \forall v\in V_2, \phi(\sigma)|_{V_2}(v)=-v$, when $\sigma$ is odd.
Therefore, the matrix of $\phi(\sigma)$ is 
\begin{equation}\(
    \begin{array}{cc}
        1 & 0\\
        0 & -1\\
    \end{array}\)
\end{equation}
when $\sigma$ is odd;  the matrix of $\phi(\sigma)$ is $I_2$ when $\sigma$ is even.

\end{solution}

\begin{problem}
   $M_{\mathrm{n}}(K):=\{(a_{i,j})_{n\times n}:a_{ij}\in K,\forall 1\leq i,j\leq n\}$.
    Let
    $$
    \begin{aligned}
    \varphi: \mathrm{GL}_n(K) & \rightarrow \mathrm{GL}\left(M_n(K)\right) \\
    A & \rightarrow \varphi(A),
    \end{aligned}
    $$
    $$
    \varphi(A) X:=A X A^{-1} ; \quad \forall X \in M_n(K) .
    $$
    (1) Illustrate $\varphi$ is the $n^2$-dimensional $K$ representation of group  $\mathrm{GL}_n(K)$;
    (2) $M_n^0(K):=\{A\in M_n(K):\tra A=0\}$. Illustrate $M_n^0(K)$ and $\langle I\rangle$ are  invariant subspaces of $\varphi$;
    (3) Prove: If $\chara K\nmid n$, then
    $
    \varphi=\varphi_{\langle I\rangle} \oplus \varphi_{M^{0}_n(K)}
    $
\end{problem}
\begin{solution}
\begin{enumerate}
    \item \begin{enumerate}
		\item $\phi(A)$ is an invertible linear transformation:
		\begin{itemize}
			\item $\phi(A)$ is linear: $\forall X, Y \in M_n(K),a,b\in K$, $\phi(A)(aX+bY)=A(aX+bY)A^{-1}=aAXA^{-1}+bAYA^{-1}=a\phi(A)(X)+b\phi(A)(Y)$
			\item $\phi(A)$ is invertible: $\phi(A)\in \GL(V)$, $\exists A^{-1}\in \GL_n(K)$ s.t. $\id=\phi(A^{-1})\circ\phi(A)(X)=A^{-1}(AXA^{-1})(A^{-1})^{-1}=X$.
		\end{itemize}
		\item $\phi$ is a group homomorphism: $\phi(AB): V\to V$, $\forall X$, $\phi(A)\circ\phi(B)(X)=A(BXB^{-1})A^{-1}=(AB)X(AB)^{-1}=\phi(AB)(X)$
	\end{enumerate}
    \item $\forall X\in M_n^0(K)$, $\tra(\phi(A)(X))=\tra(AXA^{-1})=\tra(A^{-1}AX)=\tra(X)=0$, so $\phi(A)(X)\in M^0_n(K)$ , $\phi(A)(kI)=AkIA^{-1}=kI\in \langle I\rangle$
    \item $\forall X\in  M_n^0(K)\cap \langle I\rangle$, then $X=kI$ and $\tra (X)=nk=0$.
    Since $\chara K\nmid n$, $k=0$. So $X=0$. $\forall X\in M_n(K)$, $k=\frac{1}{n}\tra(X)$, then $\tra(X-kI)=\tra(X)-nk=0$, so $X=kI+(X-kI)$, that means $M_n(K)=M_n^0(K)+\langle I\rangle$. Therefore, $M_n(K)=M_n^0(K)\oplus\langle I\rangle$
\end{enumerate}
\end{solution}
\begin{problem}
   $\calO(n):=\{A\in M_n(\R):AA^T=I_n\}$ is the set of all $n$-dimensional orthogonal matrix over $\R$. Let: 
   \begin{equation}
    \begin{aligned}
        \varphi: \calO(n) &\rightarrow \mathrm{GL}\left(M_n(\mathbb{R})\right) \\
        A &\mapsto \varphi(A), \\
    \end{aligned}
   \end{equation}
    \begin{equation}
 \varphi(A) X:=A X A^{-1}: \quad \forall X \in M_n(\mathbb{R}) 
    \end{equation}
 $M_n^{+}(\mathbb{R}):=\{A\in M_n^0(\R): A=A^T\}$, $M_n^{-}(\mathrm{R}):=\{A\in M_n^0(\R): A^T=-A\}$.
    (1) Proof: $M_n^{+}(\mathrm{R})$ and $M_n^{-}(\mathrm{R})$ are invariant spaces of $\varphi$;
    (2) Let the subrepresentation of $\varphi$ on $\langle I \rangle$,$ M_n^{+}(\mathbb{R}), M_n^{-}(\mathbb{R})$ be  $\varphi_0, \varphi_1, \varphi_2$. Proof:
    $
    \varphi=\varphi_0+\varphi_1+\varphi_2 \text {. }
    $
    (3) calculate a $\frac{1}{2} n(n-1)-$ dimensional $ \R$ representation of $\calO(n)$.
\end{problem}
\begin{solution}
\begin{enumerate}
    \item Since $\calO(n)\subset \GL_n(\R)$, $M_n^+(\R),M_n^-(\R)\subset M_n^0(\R)$, then $\forall A\in\calO(n) , X\in M_n^+(\R)(or M_n^-(\R))$, by problem 3,$\phi(A)(X)\in M_n^0(\R)$.
    Since $AA^T=I_n$, then $A^T=A^{-1}$, so
    $\forall X\in M^+_n(\R)$, $(\phi(A)(X))^T=(AXA^{-1})^T=(A^{-1})^TX^TA^T=(A^T)^{-1}XA^T=AXA^{-1}=\phi(A)(X)$, so $\phi(A)(X)\in M_n^+(\R)$. 
    $\forall X\in M^-_n(\R)$, 
    $(\phi(A)(X))^T=(AXA^{-1})^T=(A^{-1})^TX^TA^T=-(A^T)^{-1}XA^T=-AXA^{-1}=-\phi(A)(X)$, so $\phi(A)(X)\in M_n^-(\R)$. 
\item By problem $3(3)$, we get $M_n(K)=M_n^0(K)\oplus\langle I\rangle$, so we only need to proof $M_n^0(\R)=M_n^+(\R)\oplus M_n^-(\R)$. $\forall Y\in M_n^0(\R)$, $Z^+=\frac{Y+Y^T}{2},Z^-=\frac{Y-Y^T}{2}$, so $Z^+\in M_n^+(\R),Z^-\in M_n^-(\R)$ and $Y=Z^++Z^-$. 
Therefore $M_n^0(\R)=M_n^+(\R)+ M_n^-(\R)$.
$\forall X\in M_n^+(\R)\cap M_n^-(\R)$, $X^T=X=-X$, so $X=0$.
\item Let $\psi=\phi|_{M_n^+(\R)}$, since $\dim_{\R}M_n^+(\R)=\frac{1}{2}n(n-1)$, so $(\psi, M_n^+(\R))$ is a $\frac{1}{2}n(n-1)-$ dimensional representation of $\calO(n)$.
\end{enumerate}
\end{solution}

\end{document}