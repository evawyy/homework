%!Mode:: "TeX:UTF-8"
%!TEX encoding = UTF-8 Unicode
%!TEX TS-program = xelatex
\documentclass{ctexart}
\newif\ifpreface
%\prefacetrue
\input{../../../global/all}
\begin{document}
\large
\setlength{\baselineskip}{1.2em}
\ifpreface
    \input{../../../global/preface}
\newgeometry{left=2cm,right=2cm,top=2cm,bottom=2cm}
\else
\newgeometry{left=2cm,right=2cm,top=2cm,bottom=2cm}
\maketitle
\fi
%from_here_to_type
\begin{problem}
  Compute the characters of \(\sym^k V \) and \(\bigwedge^k V\).
\end{problem}

\begin{solution}
  Assume \(\{ v_i:1 \leq i \leq n\}\) is a basis of \(V\), assume \(\phi( g)\) has characters \(\{ \lambda_i:1 \leq i \leq n\}\). 
  Then \(\left\{ \sum_{\sigma \in S_k} \bigotimes_{i=1}^{k} v_{\tau\sigma ( i)}:\tau \in \mathcal{A}\right\}\) is a basis of \(\sym^k V\), where \(\mathcal{A}=\left\{f \in \fun{\{ 1,2,\cdots,k\}}{\{ 1,\cdots,n\}}: f \text{ is injection}\right\}\). 
  And \(\left\{\sum_{\sigma \in S_k} \prod_{i=1}^{k} \lambda_{\tau \sigma( i)}:\tau \in \mathcal{A}\right\}\) are it's characters. 
  For the same reason, we get \\
  \(\left\{ \sum_{\sigma \in S_k} \bigotimes_{i=1}^{k} ( -1)^{\sgn \sigma}v_{\tau\sigma ( i)}:\tau \in \mathcal{A}\right\}\) is a basis of \(\bigwedge^k V\).
  And \(\left\{\sum_{\sigma \in S_k} ( -1)^{\sgn \sigma}\prod_{i=1}^{k} \lambda_{\tau \sigma( i)}:\tau \in \mathcal{A}\right\}\) are it's characters. 
\end{solution}

\begin{problem}
  Find the decomposition of the reperesentation \(V^{\otimes n}\) using character theory.  
\end{problem}

\begin{solution}
  Assume \(V^{\otimes n}=U_1^{\oplus a_n}\oplus U_2^{\oplus b_n}\oplus V^{\oplus c_n}\). 
  And \(V \otimes V = U_1 \oplus U_2 \oplus V\). Now we try to caculate \(a_n,b_n,c_n\). 
  Since \(U_1 \otimes V \cong V\) and \(U_2 \otimes V \cong V\), we get 
  \[
    V^{\otimes n+1} = V^{\otimes n} \otimes V =( U_1^{\oplus a_n}\oplus U_2^{\oplus b_n} \oplus V^{\oplus c_n})\otimes V \cong U_1^{\oplus c_n}\oplus U_2^{\oplus c_n} \oplus V^{a_n+b_n+c_n}
  \]
  So we get 
  \[
    \begin{cases}
    a_{n+1}=c_n\\
    b_{n+1}=c_n\\
    c_{n+1}=a_n+b_n+c_n
    \end{cases}
  \]
  Then \(c_{n+2}=c_{n+1}+2c_n\). Since \(c_1=c_2=1\), we get \(c_n=\frac{2^n-( -1)^n}{3}\). 
  Thus \(a_n=b_n=\frac{2^{n-1}-( -1)^{n-1}}{3}\).
\end{solution}
\end{document}
